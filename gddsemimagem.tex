\documentclass[a4paper,draft,12pt]{article}
%\usepackage[latin1]{inputenc} %acentuacao com comandos
\usepackage[utf8x]{inputenc} %acentuacao no teclado
%\usepackage[brazil]{babel}
\usepackage{indentfirst}
\usepackage{amsfonts}
\usepackage[dvipsnames]{xcolor}
\usepackage{fancyvrb}
\usepackage{listings}
\usepackage{courier}
\usepackage{makeidx}
%-----------------------------------------------------------------------
% Tamanho da mancha
\textwidth 15cm
\textheight 24cm
%-----------------------------------------------------------------------
% Tamanho das margens
\topmargin -1cm
\oddsidemargin 7mm

%---------------------------------------------------------------------

%---------------------------------------------------------------------
% To-be-supplied
%---------------------------------------------------------------------
\newcounter{tbsnr}
\newenvironment{tbs}
{\addtocounter{tbsnr}{1}\par\bigskip \noindent\fbox{\thetbsnr}
\hspace*{\fill}\begin{minipage}{10cm}\tt}
{\end{minipage}\hspace*{\fill}\bigskip}
\newcommand{\tb}[1]{\begin{tbs}{#1}\end{tbs}}
\newcommand{\hide}[1]{{}}
%-------------------------------------------------------------------

\newtheorem{defin}{Definição}[section]
\newenvironment{Def}{\begin{defin}\em \ \ }{\end{defin}}
\newcounter{Def}

\newtheorem{exemplo}{Exemplo}[section]
\newenvironment{Explo}{\begin{exemplo}\em \ \ }{\end{exemplo}}
\newcounter{Explo}

\newtheorem{teorema}{Teorema}[section]
\newenvironment{Teor}{\begin{teorema}\em \ \ }{\end{teorema}}
\newcounter{Teor}

\newtheorem{lema}{Lema}[section]
\newenvironment{Lem}{\begin{teorema}\em \ \ }{\end{teorema}}
\newcounter{Lem}

\newcommand{\betM}[1]{{[\![}#1{]\!]_{\gM}}}
\newcommand{\gM}{\fra{M}}
\newcommand{\fra}[1]{{\mathfrak{#1}}}
%---------------------------------------------------------------------
% In proofs
%---------------------------------------------------------------------
  % PROOF ENVIRONMENT (from PT)
  % it puts a black box (or whatever \markendofproof) at the end of
  % each proof, unless the author has already done so with the
  % command \qed.
  \def\proof{\followon{Prova}}
  \def\endproof{\ifSuppressEndOfProva\global\SuppressEndOfProvafalse
  \else\xqed\fi\endfollowon}
  \def\followon#1{\trivlist\item[\hskip\labelsep{\sc#1}]}
%  \def\followon#1{\trivlist\item[\hskip\labelsep{\sc#1.}]}
  \def\endfollowon{\endtrivlist}
  % command to push black box to right of page
  \def\pushright#1{{\parfillskip=0pt\widowpenalty=10000
  \displaywidowpenalty=10000\finalhyphendemerits=0\leavevmode\unskip
  \nobreak\hfil\penalty50\hskip.2em\null\hfill{#1}\par}}
  % command to force early end of proof marker.
  %This is used if the end of proof marker would otherwise come out
  %in the wrong place.
  \def\qed{\xqed\global\SuppressEndOfProvatrue}
  \newif\ifSuppressEndOfProva\SuppressEndOfProvafalse
  \def\xqed{\pushright\markendofprova}
  \def\markendofprova{\rule{1.3217ex}{1.3217ex}}
%---------------------------------------------------------------------

%-----------------------------------------------------------------------
\begin{document}
%-----------------------------------------------------------------------
\thispagestyle{empty}
%-----------------------------------------------------------------------
% Espaco entre as linhas
%\baselineskip 6mm %espaco um
%\baselineskip 7mm
\baselineskip 7.5mm %espaco um e meio
%\baselineskip 9mm %espaco duplo
%-----------------------------------------------------------------------

\title{\bf Soap Bubbles Attack \\ \vspace{3mm}\large Game Design Document}
\author{\emph{Euller Macena e Mariana Ferreira}\\ \small LabJogos, Turma AA}
\date{ }

%---------------------------------------------------------------------
\maketitle
%---------------------------------------------------------------------

%\begin{abstract}
%High concept
%\end{abstract}
%
%\bigskip
%\tableofcontents

\vfill

\bigskip
\bigskip

\begin{center}
{\bf\large High Concept}
\bigskip

{\parbox{13cm}{\begin{quotation}\large O jogo se passa em um zoológico com animais estourando bolhas de sabão feitas por crianças.\end{quotation}}}
\end{center}

\bigskip
\bigskip

\vfill

%---------------------------------------------------------------------
\section{História}
%---------------------------------------------------------------------

Você está tranquilamente passeando pelo zoológico quando crianças começam a brincar de soprar bolinhas de sabão nos seus olhos! Com seus poderes de falar com animais você os convoca para te ajudar a se proteger.

\vfill

%---------------------------------------------------------------------
\section{Gameplay}
%---------------------------------------------------------------------

Bolhas de sabão de diferentes cores surgem num canto da tela e seguem por um caminho. Você deve apertar W, S ou D para escolher qual animal vai atirar para estourar as bolhas antes que alcancem seus olhos. Existem três tipos diferentes de bolhas, transparente para nível 1, verde para nível 2 e laranja para nível 3. Quando estouradas sua pontuação aumenta de acordo com o nível da bolha estourada. Você inicia o jogo com 50 de vida e a cada bolha que acerta seu olho a vida vai diminuindo.

\bigskip
\pagebreak
%---------------------------------------------------------------------
\section{Interface}
%---------------------------------------------------------------------
Soap Bubbles Attack possui duas telas, a inicial e a tela principal, onde o jogo realmente acontece.

\vspace{20cm}

%---------------------------------------------------------------------
\section{Controles}
%---------------------------------------------------------------------
Soap Bubbles Attack precisa apenas dos seguintes controles:
\begin{itemize}
\item W para o tiro do pássaro
\item S para o tiro da llhama
\item A para o tiro do elefante
\end{itemize}

%---------------------------------------------------------------------
%\section{Áudio}
%%---------------------------------------------------------------------
%A experiência de jogo de Soap Bubbles Attack é valorizada pelo áudio nos momento chaves do jogo:
%\begin{itemize}
%\item Música de fundo.
%\item Sons de bolhas estourando cada vez que são acertadas.
%\item Som ao final para \emph{game over}.
%\end{itemize}

%---------------------------------------------------------------------
\section{Arte Conceito}
%---------------------------------------------------------------------

\vspace{8cm}
Esses são os animais que te ajudam a se defender das bolhas de sabão nos olhos.
%---------------------------------------------------------------------
\section{Referências}
%---------------------------------------------------------------------
Alguns jogos que nos inspiraram foram:

\begin{itemize}
\vspace{8cm}
\item[] Bloons Tower Defense
\vspace{8cm}
\item[] Fieldrunners 2
\end{itemize}

%-----------------------------------------------------------------------
%\begin{thebibliography}{5}

%\bibitem{dolittle} Dr. Dolittle

%\end{thebibliography}

%\tb{falta colocar referências para Dr. Dolittle}

%-----------------------------------------------------------------------
\end{document}
%-----------------------------------------------------------------------